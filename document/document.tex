\documentclass[
        paper=a4,       % Seitenformat
        fontsize=11pt,  % Schriftgröße
        oneside,        % einseitig
        headsepline,    % Trennlinie für die Kopfzeile
        notitlepage     % keine extra Titelseite
]{scrartcl}             % KOMA-Script Article
%------------------------------------------------------------------------

\usepackage[automark]{scrlayer-scrpage}     % Seiten-Stil für scrartcl
\usepackage[top=25mm]{geometry}  		    % Oberer Rand 25mm Einrückung
\usepackage[utf8]{inputenc}                 % Eingabekodierungen
\usepackage[T1]{fontenc}                    % Eingabekodierungen
\usepackage[english,ngerman]{babel}         % Mehrsprachenumgebung, Hauptsprache Deutsch
\usepackage{setspace}                       % Zeilenabstand
\usepackage{latexsym}                       % Latex-Symbole
\usepackage{amsfonts,amssymb,amstext}       % Mathematische Formeln
\usepackage{bbm}                            % bbm Schriftart
\usepackage{graphicx}                       % Abbildungen einbinden
\usepackage{listings}						% Programmcode einbingen
\usepackage{xcolor}						    % Farben
\usepackage{enumitem}                       % Aufzählungen
\usepackage{mathptmx}                       % Times New Roman Schriftart
\usepackage{forest}
\usepackage{silence}
\usepackage{amsmath}
\usepackage{amsthm}
\usepackage{subcaption}
\usepackage{xparse}
\usepackage{csquotes}

\pagestyle{scrheadings}					    % Kopfzeilen nach scr-Standard		

%-----------------------------------------------------------------------------------
%%% DOKUMENT-KOPF %%%
%-----------------------------------------------------------------------------------

% Definition von Variablen für Name, Titel und Übungsgruppe
\makeatletter
\gdef\autor#1{\gdef\@autor{#1}}
\def\@autor{Kein Autor}
\gdef\titel#1{\gdef\@titel{#1}}
\def\@titel{Kein Titel}
\gdef\gruppe#1{\gdef\@gruppe{#1}}
\def\@titel{Keine Gruppe}

\parindent0em

% Horizontale Linie mit Abstand zu den Seitenrändern
\newcommand{\ownline}{
    \vspace{.7em}
    \hrule
    \vspace{.7em}
} 

% Aufgabenkopf
\newcommand{\aufgabe}[1]{
    \vspace*{0.5cm}
    \large{
        \textbf{Aufgabe #1}
    }
    \normalsize\vspace*{0.5cm}
}

% Befehl zum Erzeugen des Kopfes
\newcommand{\kopf}{
    \ownline
    \begin{center}
        {\LARGE \textbf{\@titel}} \\
        \mbox{} \\
        {\large \@autor} \\
    \end{center}
    \ownline						
}
\makeatother

% Inline code
\definecolor{una-green}{HTML}{489324}
\NewDocumentCommand{\codeword}{v}{\texttt{\textcolor{una-green}{#1}}}

%-----------------------------------------------------------------------------------

% Hier eingetragene Daten erscheinen im Dokumentkopf
\autor{Julian Hoffmann, Christian Spreng}
\titel{Python Software Project Planing Document}

\begin{document}

\pagestyle{plain}

\kopf

\section{Game}
\begin{itemize}
    \item Sudoku
\end{itemize}

\section{Role Distribution}
\begin{itemize}
    \item Julian: UI \& Interaction \& AI
    \item Christian: Gamelogic \& AI
\end{itemize}

\section{Implementation}
\textit{Because of space constraints we replaced the keyword} \codeword{self} \textit{with} \codeword{s}.\par
\subsection{Game Class}
The core class representing a Sudoku game instance. Handles the game board, rules, and overall game flow.
A gamesession contains the initial boardvalues, the placed values and the difficulty.\par
\textcolor{white}{text}\par

\begin{tabular}{|p{6cm}|p{5cm}|p{4cm}|}
    \hline
    \textbf{Method} & \textbf{Input} & \textbf{Output} \\
    \hline
    \codeword{__init__(s, difficulty)} & Difficulty level. & None \\
    \hline
    \codeword{load_board(s, filename)} & Filename of the board. & None \\
    \hline
    \codeword{generate_board(s, difficulty)} & Difficulty level. & None \\
    \hline
    \codeword{is_valid(s, row, col, value)} & Row, column, and value. & \codeword{True} if move is valid. \\
    \hline
    \codeword{is_game_over(s)} & None & \codeword{True} if puzzle is solved. \\
    \hline
    \codeword{save_game(s, filename)} & Filename to save the game. & None \\
    \hline
    \codeword{get_board(s)} & None & The current game board. \\
    \hline
    \codeword{update_val(s, row, col, value)} & Row, column, and value. & None \\
    \hline
    \codeword{delete_all_values(s)} & None & None \\
    \hline
\end{tabular}

\subsection{Player Class}
Represents a player of the game. Stores player information and high scores.\par
\textcolor{white}{text}\par

\begin{tabular}{|p{6cm}|p{4cm}|p{5cm}|}
    \hline
    \textbf{Method} & \textbf{Input} & \textbf{Output} \\
    \hline
    \codeword{__init__(s, name, pwd)} & Name and password. & 0 (Error), 1(Success), 2(New) \\
    \hline
    \codeword{get_highscore(s, diffic)} & Difficulty level. & User's high score for difficulty. \\
    \hline
    \codeword{update_score(s, diffic, time)} & Difficulty and time. & None. \\
    \hline
\end{tabular}

\subsection{UserInterface Class}
Handles user interaction, displaying the game board, getting input, and providing feedback.\par
\textcolor{white}{text}\par

\begin{tabular}{|p{5cm}|p{5cm}|p{5cm}|}
    \hline
    \textbf{Method} & \textbf{Input} & \textbf{Output} \\
    \hline
    \codeword{__init__(s, game)} & Reference to the \codeword{Game} object. & None \\
    \hline
    \codeword{display_board(s)} & None & None \\
    \hline
    \codeword{get_move(s)} & None & A tuple of row, column, value. \\
    \hline
    \codeword{display_message(s, msg)} & Message string. & None \\
    \hline
    \codeword{display_highscores(s)} & None & None \\
    \hline
    \codeword{choose_difficulty(s)} & None & The selected difficulty level. \\
    \hline
\end{tabular}

\subsection{HighscoreManager Class}
Manages the overall high score list for all users.\par
\textcolor{white}{text}\par

\begin{tabular}{|p{6cm}|p{6cm}|p{3cm}|}
    \hline
    \textbf{Method} & \textbf{Input} & \textbf{Output} \\
    \hline
    \codeword{load_scores(s, file)} & Filename of the highscore file. & None \\
    \hline
    \codeword{save_scores(s, file)} & Filename to save high scores. & None \\
    \hline
    \codeword{add_score(s, p, diffic, time)} & \codeword{Player} obj, difficulty level, and time. & None \\
    \hline
    \codeword{get_top_scores(s, diffic, n)} & Difficulty \& \(n\) of top scores to retrieve. & High scores list. \\
    \hline
\end{tabular}

\subsection{AIPlayer Class}
Implements an AI player that interacts with the game like a human.\par
\textcolor{white}{text}\par

\begin{tabular}{|p{5cm}|p{5cm}|p{5cm}|}
    \hline
    \textbf{Method} & \textbf{Input} & \textbf{Output} \\
    \hline
    \codeword{__init__(s, game, diffic)} & Ref to \codeword{Game} obj and difficulty. & None \\
    \hline
    \codeword{make_move(s)} & None & Tuple of row, column, value of AI's move. \\
    \hline
    \end{tabular}

\subsection{Additional Functions}
\begin{tabular}{|p{5cm}|p{5cm}|p{5cm}|}
    \hline
    \textbf{Method} & \textbf{Input} & \textbf{Output} \\
    \hline
    \codeword{main()} & None & None \\
    \hline
    \codeword{load_game(filename)} & Filename of the game file. & A \codeword{Game} object. \\
    \hline
    \codeword{instructions()} & None & None \\
    \hline
    \end{tabular}

\subsection{Difficulty Levels}
\begin{tabular}{|p{5cm}|p{10cm}|}
    \hline
    \textbf{Difficulty} & \textbf{Description} \\
    \hline
    Easy & Many given numbers, easy to solve. \\
    \hline
    Medium & Fewer given numbers, slightly more challenging. \\
    \hline
    Hard & Only a few given numbers, challenging. \\
    \hline
    Expert & Extremely few given numbers, very difficult. \\
    \hline
    \end{tabular}

\subsection{UI Enhancements}
\begin{itemize}
    \item Usage of the \codeword{curses} for improved console display (colored cells, highlighting, \dots)
\end{itemize}

\end{document}
